\documentclass[11pt]{article}

\usepackage{amsmath}
\usepackage{textcomp}
\usepackage[top=0.8in, bottom=0.8in, left=0.8in, right=0.8in]{geometry}
% add other packages here

% put your group number and names in the author field
\title{\bf Exercise 5: An Auctioning Agent for the Pickup and Delivery Problem}
\author{Group 24: Benvenuti, Bronner}

\begin{document}
\maketitle

\section{Bidding strategy}
% describe in details your bidding strategy. Also, focus on answering the following questions:
% - do you consider the probability distribution of the tasks in defining your strategy? How do you speculate about the future tasks that might be auctions?
% - how do you use the feedback from the previous auctions to derive information about the other competitors?
% - how do you combine all the information from the probability distribution of the tasks, the history and the planner to compute bids?
Our strategy is striking by its simplicity. The goal is to take as many tasks as possible and then the price to move the tasks will decrease relative to the number of tasks, thanks to the optimal path finder (Strategy inspired by UBERPOOL). For this strategy to work we need to bid aggressively at the beginning, to be sure to take the first tasks. Then, if the concurrence wins a bid, we directly get cheaper for the next client.

The value of the bid is in function of the following parameters
\begin{itemize}
  \item Distance: The distance between pickup and deliver cities.
  \item CostKm: The cost per kilometer of the cheapest truck that can carry this task.
  \item Greed: A parameter that will increase if last bid has been won, decrease otherwise. 
\end{itemize}

The asked prize for the bid is therefore: $Distance * CostKm * Greed$. 

The parameter $Greed$ has the following comportment.
\begin{itemize}
  \item If we won last bid: $Greed = Greed * 1.1$
  \item If we lost the last bid : $Greed = Greed * 0.8$
  \item Greed cannot be greater than 1 or smaller than 0.1.
\end{itemize}
The initialization of Greed tells how much we want to have the first tasks. 

It has been decided that the agent will not try to infer information about the oppenent. There is no way to know its strategy in only two parties. 

\section{Results}
% in this section, you describe several results from the experiments with your auctioning agent

\subsection{Experiment 1: Comparisons with dummy agents}
% in this experiment you observe how the results depends on the number of tasks auctioned. You compare with some dummy agents and potentially several versions of your agent (with different internal parameter values). 
Two dummy agents, named "Honest and Random", were created, both will compute the additional cost of transporting the new task. This difference is calculated by computing the costs to transport the N-1 tasks and N task (computed with our centralized algorithm). The two agents will then make the following bids:
\begin{itemize}
  \item Honest: $bid = price + 10$
  \item Random: $bid = price * unif(1,2)$
\end{itemize}

\subsubsection{Setting}
% you describe how you perform the experiment, the environment and description of the agents you compare withe 
The tests were all made in the England map, we test for different number of randomly distributed tasks (5, 10, 20 and 30). We also test whenever we have better, same or worst trucks. 
The characteristics of the different trucks are the following:
\begin{itemize}
  \item Efficient trucks: capacity of 30 and cost-per-km of 5.
  \item Inefficient trucks: capacity of 20 and cost-per-km of 7.
\end{itemize}

The results are given in Tables \ref{tab:1} and \ref{tab:2}.

\subsubsection{Observations}
% you describe the experimental results and the conclusions you inferred from these results
\begin{table}
  \begin{center}
    \begin{tabular}{ | l | l | l | }
    \hline
    Nb tasks & Opponent & Result   \\ \hline
    5 & random & Victory \\ \hline
    5 & honest & Victory \\ \hline
    10 & random & Victory \\ \hline
    10 & honest & Victory \\ \hline
    20 & random & Victory \\ \hline
    20 & honest & Victory \\ \hline
    30 & random & Victory \\ \hline
    30 & honest & Victory \\ \hline

    
    \end{tabular}
    \caption{Results of our algorithm over the dummies when using the same trucks, (the efficient ones).}
    \label{tab:1}
\end{center}
\end{table}

\begin{table}
  \begin{center}
    \begin{tabular}{ | l | l | l | }
    \hline
    Nb tasks & Opponent & Result   \\ \hline
    5 & random & Victory \\ \hline
    5 & honest & Victory \\ \hline
    10 & random & Victory \\ \hline
    10 & honest & Victory \\ \hline
    20 & random & Victory \\ \hline
    20 & honest & Victory \\ \hline
    30 & random & Victory \\ \hline
    30 & honest & Victory \\ \hline
    
    \end{tabular}
    \caption{Results of our algorithm over the dummies when our agent used the inefficient trucks.}
    \label{tab:2}
\end{center}
\end{table}

Following the results of Table \ref{tab:1}, it doesn't seem necessary to check when our agent has better trucks. The results show in Table \ref{tab:2} seems to indicate that the agent could win in subobtimal situations and that although simple it is efficient.
%\vdots

%\subsection{Experiment n}
% other experiments you would like to present (for example, varying the internal parameter values)

%\subsubsection{Setting}

%\subsubsection{Observations}

\end{document}
